%\input logo

%\vspace*{-3cm}

%\begin{figure}[h]
%\leavevmode
%\begin{minipage}[t]{\textwidth}
%\includegraphics[1cm,1cm][3cm,3cm]{logo-ufrpe.bmp}
%\end{minipage}
%\end{figure}



\vspace*{-2cm}
{\bf
\begin{center}
{\large
\hspace*{0cm}Universidade de São Paulo} \\
\hspace*{0cm}Escola Politécnica \\
\hspace*{0cm}Curso de Engenharia de Computação  \\
\end{center}}
% se você souber meter o logo da POLI aqui, ficaria legal...
\vspace{4.0cm}
\noindent
\begin{center}
{\Large \bf Ressonância no sistema massa-mola} \\[3cm]
{\Large Adilson Torres Gregório de Souza, adilson.souza@usp.br}\\[6mm]
{\Large Jhonata Antunes, jhonata.antunes@usp.br}\\[6mm]
{\Large .%MAP3122 - Métodos Numéricos
}\\[3.0cm]
\end{center}

{\raggedleft
\begin{minipage}[t]{8.0cm}
\setlength{\baselineskip}{0.25in}
RELATÓRIO apresentado ao Professor Alexandre Roma do MAP/IME-USP 
como atividade da disciplina MAP3122 - Métodos Numéricos.
\end{minipage}\\[2cm]}
% o logo também ficaria legal aqui legal...
\vspace{2cm}
{\center São Paulo - SP \\[3mm]
10/04/2016 \\}


\newpage